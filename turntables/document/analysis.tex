\section{Analysis}

We investigate possible approaches in order to bring up persistance of data for
(possibly small) applications. Persistance in applications may be required to 
be created once. On the other hand, if the application is to be updated over
different time intervals, we wish to have some sort of mechanism to manage the
versioning. So we identify two sorts of transactions: monolithic, and
sequential transactions.

\subsection{Monolithic Transaction Creation}

By a monolithic transaction, we want to contrast between something that occurs
once, and is responsible to bring up the whole database in one go, in contrast
to a sequential upgrade, where small steps are performed to reach the desired
outcome.

A monolithic build would require all of the required schemas to be in one file
(or at least something in standalone format).

\subsubsection{Benefits}

One noted benefit to this approach is that building the the database with only
one final schema would be much quicker than to send the database small requests
each time there needs to be an alteration.

\subsubsection{Cons}

Maintainable versioning can not be done this way. For instance, if we were to 
change application from version 1.0, to version 2.0, and the database needed
changing, this method would suggest that everything would be teared down, and
put back up. If there is no data that we care about that is being persisted,
then there is no harm done, and this is totally a feasable approach. If however
the data currently needs to be persisted from one version to the other, or even
altered for the next version, this approach would not be good.

\subsection{Sequential Transaction Creation}

By sequential transaction creation, we wish to contrast against the
aforementioned method of schema creation on the database. That is, sequential
transaction creation is the collection of small steps in order to bring up 
the whole database.

\subsubsection{Benefits}

Versioning would be possible this way. Each revision to be executed in order to
alter the database is able not only add and remove whole tables, but to drop,
add columns, change their types, seed data to tables, and all sorts of other
goodness.

\subsubsection{Cons}

As opposed in committing everything in one bang, applying alterations to the 
database this way would be slower.

\subsection{Best of two worlds}

Sacrifice of performance can be minimized to an extent; however this would be
to the mercy of the developers. Monolithic transactions in combination with
sequential may be used in order to achieve a situation that is somewhat better
performant than a pure sequential creation of the database. 

We can create fixed points that bundles a collection of sequential
transactions. This grouping would denote that version `A' would require
sequences `1-4', and version `B' would require sequences `5-10'. 

Therefore, instead of doing a sequential create, we can omitt everything
previous to version `B' by having a fixed point that bundles all the
transactions from version `A' in one, and then perform the last sequential
operations that are required to upgrade to version `B'.

\subsection{Additional Requirements}

In order to make this mechanism work, we also need an additional part. We wish
to somehow store the current version of the database somewhere in order to 
query about the transactions that are required to be fired off. 

The best in this case would be to store a table inside the database that would
contain such information. This would allow us to easily retrieve information
about the current version of the database, along with a description, and a date
the transaction was performed in case the software product that is using this
concept needs to have failures or otherwise evaluated.

The descriptions that are to be added for each revision, in one of these 
entites, could be contained within the transactions that are to be done to the
database. This allows for more information to be extracted if things go wrong.

