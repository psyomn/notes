\documentclass[12pt,twoside]{article}

\title{Nous: A simple monitoring system}
\author{Simon Symeonidis}
\date{Fri Feb  2 11:01:51 EST 2018}
\usepackage[margin=1in]{geometry}
\usepackage{listings}
\usepackage{color}
\usepackage[colorlinks]{hyperref}
\usepackage{graphicx}
\usepackage{grffile} % added to understand *.png
\usepackage[export]{adjustbox}
\usepackage[labelfont=bf]{caption}
\DeclareCaptionFont{white}{\color{white}}
\DeclareCaptionFormat{listing}{\colorbox{psyan}{\parbox{\textwidth}{#1#2#3}}}
\captionsetup[lstlisting]{format=listing,labelfont={white,bf},textfont={white}}

%%% Macros
%%%
%%% Macro shorthand to add figures
%%%

\newcommand{\psyfig}[3]{%
    \begin{figure}[h!]
    \centering
    \includegraphics[max width=\linewidth, keepaspectratio]{#1}
    \caption{#2}
    \label{fig:#3}
    \end{figure}}

    %% TODO LEGACY: need to fix other document parts using this
\newcommand{\psyfigd}[3]{%
    \begin{figure}[h!]
    \centering
    \includegraphics[max width=\linewidth, keepaspectratio]{#1}
    \caption{#2}
    \label{fig:#3}
    \end{figure}}


\input{psy-macros/psy-rcode}
%%
%% Add posibility to boldface first word of item
%%

\newcommand{\itembf}[1]{%
  \item \textbf{#1}
  }


\pagenumbering{arabic}

\begin{document}

\maketitle
\tableofcontents
\listoffigures
\lstlistoflistings

%%% Not actually a macro - forced to declare here due to lstset et al.
\input{psy-macros/psy-listing}

\section{Summary}
Ensuring that large systems are working well together can be
very messy. As a result simple problems that could be avoided
get neglected and build up to problems that can result in
failures.

We treat systems as black boxes, and view them connected by
the data flow that enters and exits each of them. These systems as
black boxes are then monitored with a publish or subscribe method to a
central repository, that stores and alerts about infractions on
defined contracts.

The contracts are defined with a simple DSL, which declare certain
endpoints of systems, where they can be queried about their
status. For the implementation of \textit{NOUS} this is done via
\textit{JSON REST} endpoints, being queried and compared against each
other.

\section{Contracts}
Should be human readable SLAs. Human readable documentation: enforce
documetation Conditions on served json from different endpoint. This
is done with the `WITH' keyword, which is not optional. We see a
DSL/Sketch of this contracts as such:

%% TODO: What would be cases against dynamic loading in this context?
%% ie our contracts could just be erlang files that get dynamically
%% loaded and compiled.

\begin{lstlisting}[language=nouscontract,caption='Sample Contract']
-- NOTE: all of the endpoints should be returning JSON. We are
-- assuming JSON everywhere, though there could middleware that deals
-- with other formats.
--
-- TODO: do we care about invariants?
-- Collect different variables from different endpoints that
-- eventually will be needed to form the contract
--
-- TODO: what kind of access do we want for the keys?
-- TODO: what about array index access?
FROM www.a.com/status PORT 1234
 GET .key1 AS a_key1,
     .key2 AS a_key2,
     .keyx.innerkey AS a_innerkey;

FROM www.b.com/otherplace/status
PORT 80
 GET .key1 AS b_key1,
     .key2 AS b_key2,
     .keyx.innerkey AS b_innerkey;

FROM 123.32.21.23/rofl/hahaha/status&derp=json
PORT 9001
 GET .key1 AS c_key1,
     .key2 AS c_key2,
     .keyx.innerkey AS c_innerkey;

ENSURE                  -- example contract
  a_key1 =  b_key2 AND  -- logically this means nothing, just
  b_key1 <  a_key2 OR   -- sketching to see what I care about
  c_key1 >  a_key3 AND  -- dsl-wise
  c_key1 >= a_key3 AND  -- ie: want and, or logical ops, and
  c_key1 <= a_key2 OR   -- <, >, <=,  >=, <> (not)
  c_key3 <> b_key3

WITH
  '
  This is some text that will go here and should go here
  and support multiline stuff. Also it should describe what
  went wrong. For example, this contract would say that
  "a_key1 must be equal to b_key2, because we have some
  client that required the system a certain way."
  ';
\end{lstlisting}

\section{Layout of Platform}
Registry like platform as it connects a bunch of services
together. Publish or subscribe architecture will make it possible to
choose whether applications are

\section{Security}
Should there be methods of authentication, or do we rely on firewalls
setup by the box?

\section{History}
Should we worry about storing snapshots of things over time? By
doing this, we could have a easy way for querying things.

\paragraph{Optimistic Storing}
Store everything (n*m where n is services, and m are number log
from intervals) (ie a lot). This, though wasteful, can give a
better view of what is happening to the system over a span of
time.

\paragraph{Pessimistic Storing}
Store only infractions. This will store less things but on the
other hand will not allow for an overall view as the above.

\section{Storage Method}
Snapshots of infractions (probably json?). Don't need complicated
store. Maybe even Mnesia would be good for this use case.

\section{Runtime Requirements}
Being able to dynamically ship contracts. This could be done
through networked file transfer for example. ie: have a fodler
`contracts/`, that gets poppulated with contracts shipped by some
technique.

\subsection{Reliability}
What hard promises can this service provide?
What happens if this service crashes?

\section{Alerting Mechanisms}
Methods of nofifications for when something goes bad.
\begin{itemize}
  \item Send an email (and eventually maybe even fine grain to poke
    people responsible for failures).
\end{itemize}

\section{Questions and Notes}
\begin{itemize}
  \item{Service Disovery}: This might be a little problematic: how can
    we know where services with different statuses exist?
  \item{YECC}: (erlang parser generator) to make parser that
    understands contract dsl.
\end{itemize}

\end{document}
