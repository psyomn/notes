\definecolor{blue}{RGB}{0,0,255}
\definecolor{green}{RGB}{0,100,0}
\definecolor{psyan}{RGB}{0,104,222}
\definecolor{lightpsyan}{RGB}{185,239,254}
\definecolor{red}{RGB}{255,0,0}

\lstdefinelanguage{Rust}{
morekeywords={struct,fn,%
do,else,false,%
for,if,in,import,match,mod,%
Some,None,crate,%
pub,return,%
self,true,%
impl,let,while,%
i8,u8,i16,u16,i32,u32,i64,u64,usize},
otherkeywords={->},
sensitive=true,
morecomment=[l]{//},
morecomment=[n]{/*}{*/},
morestring=[b]",
morestring=[b]',
morestring=[b]"""
}

\lstset{ %
  basicstyle=\ttfamily\scriptsize,        % the size of the fonts that are used for the code
  breakatwhitespace=false,         % sets if automatic breaks should only happen at whitespace
  breaklines=true,                 % sets automatic line breaking
  captionpos=t,                    % sets the caption-position to bottom
  commentstyle=\color{green},      % comment style
  deletekeywords={...},            % if you want to delete keywords from the given language
  escapeinside={\%*}{*)},          % if you want to add LaTeX within your code
  extendedchars=true,              % lets you use non-ASCII characters; for 8-bits encodings only, does not work with UTF-8
  frame=tb,                         % adds a frame around the code
  keepspaces=true,                 % keeps spaces in text, useful for keeping indentation of code (possibly needs columns=flexible)
  keywordstyle=\color{psyan},      % keyword style
  language=Ada,                   % the language of the code
  morekeywords={*,...},            % if you want to add more keywords to the set
  numbers=left,                  % where to put the line-numbers; possible values are (none, left, right)
  numbersep=5pt,                 % how far the line-numbers are from the code
  numberstyle=\tiny\color{black}, % the style that is used for the line-numbers
  rulecolor=\color{black},         % if not set, the frame-color may be changed on line-breaks within not-black text (e.g. comments (green here))
  showspaces=false,                % show spaces everywhere adding particular underscores; it overrides 'showstringspaces'
  showstringspaces=false,          % underline spaces within strings only
  showtabs=false,                  % show tabs within strings adding particular underscores
  stepnumber=1,                    % the step between two line-numbers. If it's 1, each line will be numbered
  stringstyle=\color{red},         % string literal style
  tabsize=2,                       % sets default tabsize to 2 spaces
  framexbottommargin=5pt,
  framextopmargin=5pt,
  title=\lstname                   % show the filename of files included with \lstinputlisting; also try caption instead of title
}
